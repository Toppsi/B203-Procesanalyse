\chapter{Læreprocesser}
I det kommende afsnit vil læreprocessen for projektet og de relevante kurser blive beskrevet, vurderet og reflekteret.

\section{Beskrivelse}
I dette afsnit vil anvendte læringsmetoder og -processer blive beskrevet.

\subsection{Læringsmetode}
Vi afholdt et indledende møde, da vi startede i gruppen. Her præsenterede vi os selv og hvilke læringsmetoder, der virkede bedst for hvert enkelt gruppemedlem. Derudover havde vi en forventningsafstemning, hvor vi beskrev vores individuelle ambitionsniveau. De tre overordnede emner; mål for projektet og forelæsninger, foretrukne arbejdsformer samt forventninger til hjemmearbejde. \\

\subsection{Læreproces ved Kurser}
Vi havde i gruppen forskellige udgangspunkter til kurserne. Af helbredsmæssige årsager var enkelte medlemmer forhindret i at deltage til forelæsninger. Det blev derfor aftalt, at opgaveregning skulle være obligatorisk for alle, men at forelæsningerne var valgfrie. Det var påkrævet, at vi mødte forberedt op til opgaveregning. \\
Vi benyttede os af tavleregning i starten af perioden med CAL, hvor vi skiftedes til at gå til tavlen. I FTP- og FVM-kurset blev opgaverne gennemgået på gruppen, hvorefter resultatet blev tjekket, og i tilfælde af fejl blev opgaven diskuteret. Ved FTP-kurset var der praktiske øvelser, hvor gruppen var delt op i to, efterfulgt af en fremlæggelse af opgaverne. 
  
\subsection{Læreproces ved Projektarbejde}
Igennem projektperioden har der ikke været optimal fokus på hinandens læreprocesser. Desuden har grupperumsarbejde været udfordrende på nogle punkter, da gruppens medlemmer arbejder bedst på forskellige måder. Gruppen har arbejdet i roterende undergrupper.

 
\section{Vurdering}
Her opstilles de beskrevne emner i positive og negative erfaringer.
\begin{table}[h]
	%\caption{This Is a Table\label{LABEL}}
	\begin{tabular}{|p{3cm}|p{5cm}|p{5cm}|}
		\hline
		&\textbf{Positive erfaringer}&\textbf{Negative erfaringer}\\ \hline
	\textbf{Læringsmetode}	& Grundig forventningsafstemning. \newline Forventningsafstemning blev ikke taget lige seriøst af alle gruppemedlemmerne. \newline Forventningsafstemningen: ikke udnyttet fulde potentiale. \\ \hline
	\textbf{Læreproces ved kurser} & Obligatorisk opgaveregning. \newline  Opgaveregning på tavlen. & \\ \hline
	\textbf{Læreproces ved projektarbejde} & Fordeling af projektarbejde i undergrupper. \newline Rotering af arbejdsopgaver. & Optimale læreprocesser og -metoder blev ikke udnyttet i tilstrækkeligt omfang.\\ \hline
		
	\end{tabular}
\end{table}

\section{Refleksion}
Der bliver i det følgende afsnit reflekteret over de pågældende emner.

\subsection{Læringsmetode}
Alle ytrede deres forskellige baggrunde og erfaringer, hvilket gjorde forventningsafstemningen optimal. Dette gav os et indblik i, hvordan de enkelte lærte bedst, men der blev ikke taget aktivt hensyn til dette. Forventningsafstemningen gav dog en god diskussion omkring vores individuelle arbejdsmetoder og eventuelle emner, som gruppens medlemmer ville have vendt.

\subsection{Læreproces ved kurser}
Det fungerede godt at vi arbejdede med opgaverne på tavlen i CAL, da metoden gav en god forståelse af opgaverne. Herved fik alle muligheden for at regne opgaver med hjælp fra gruppen. Da nogle medlemmer regnede hurtigere end andre, blev denne metode fravalgt senere på semestret. Vi burde dog bibeholde metoden, da de hurtige i CAL nødvendigvis ikke er de hurtige i FTP, og man derved bør hjælpe hinanden. \\
% I FTP-kurserne var det et problem, at alle opgaverne ikke blev taget lige seriøst grundet tidspres fra projektet. Der var delte meninger om, hvordan der skulle tages hånd om problemet. Dette resulterede i, at opgaverne enten måtte laves individuelt eller udskydes til eksamensforberedelse.
 
\subsection{Læreproces ved projektarbejde}
Det manglende fokus på det enkelte gruppemedlems foretrukne arbejds- og læringsstil har resulteret et uproduktivt arbejdsmiljø. Dette var også en af grundende til, at sluttiden for arbejdsdagen ikke blev overholdt i starten af projektperioden. Det var en udfordring for gruppen at finde en balance igennem hele forløbet. \\
Opdelingen af gruppen og rotering i undergrupper til retteprocessen har gavnet vidensdelingen. Derudover gav roteringen et nyt perspektiv på opgaven, hvilket sikrede rapporten blev et produkt af vores fælles indsats. \\
I FTP-kursernes praktiske øvelser fik vi erfaring med udstyret, som vi skulle benytte til vores forsøg i projektet. Dette var relevant for gruppen, da vi derved fik erfaring til at udføre projektets forsøg.

\section{Syntese}
I P$3$ vil det være en fordel at være mere opmærksom på gruppens individuelle arbejdsprocesser for at optimere samarbejdet. Vi vil fortsat arbejde i roterende undergrupper men fremadrettet skifte arbejdspartner oftere.
Vi vil fokusere mere på at arbejde sammen til opgaverne i kurserne. Hermed menes der, at de hurtige i det pågældende kursus skal være mere tålmodige. Dette kunne med fordel skrives ind i samarbejdsaftalen.