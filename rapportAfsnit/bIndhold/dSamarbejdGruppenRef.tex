\section{Refleksion}
I dette afsnit vil de enkelte faktorer for gruppens samarbejde med hinanden blive reflekteret.

\subsection{Samarbejdsaftale}
At have samarbejdsaftalen på skrift var positivt, da alle gruppemedlemmer derved tog aftalen alvorligt. Derudover gav det enighed om, hvilke regler gruppearbejdet skulle overholde. Dette førte til et bedre samarbejde, da alle gruppemedlemmer var enige om og indforstået med kriterierne for et godt samarbejde. \\
Opdelingen af samarbejdsaftalen var et kompromis i gruppen og havde både positive og negative konsekvenser. Selve samarbejdsaftalen blev kort og præcis, hvorved den blev lettere at overholde. Dokumentet "Tips og Tricks" blev dog taget mindre seriøst, da det blev tolket som forslag og ikke retningslinjer af nogle gruppemedlemmer. Opdelingen medførte, at samarbejdsaftalen og "Tips og Tricks" blev svære at overskue.

\subsection{Gruppemøder og Gruppearbejde}
Den ustrukturerede mødetid i starten af projektperioden havde en negativ indflydelse på de enkeltes arbejdsindsats, da flere foretog overspringshandlinger. Den senere fastsatte sluttid gav således overblik over samarbejdet, da det var muligt for alle gruppemedlemmer at følge med i hinandens arbejde. Dette gav tryghed for nogle gruppemedlemmer, da det var muligt at se projektets progressive forløb. \\ %Det er dog vanskeligt at holde fuld koncentration hele dagen, hvilket medførte et splittet og opdelt arbejdsflow.
Opgaveregningen for semestres tilhørende kurser blev gjort obligatorisk for alle gruppemedlemmer, da fremmødet hertil var mangelfuldt. Det lille fremmøde virkede demotiverende og vanskeliggjorde opgaveregningen, hvormed gruppen blev enige om at gøre den obligatorisk. Hermed tog gruppen et fælles ansvar for indlæring, hvilket, udover at forbedre forståelsen for kurserne også, gav bedre sammenhold i gruppen. \\ %Denne regel tog dog ikke hensyn til gruppemedlemmernes forskellige foretrukne arbejdsmetoder, hvilket kan have hæmmet indlæringen ved nogen. Gruppen forsøgte at tage højde for dette, da fremmøde til forelæsninger ikke blev gjort obligatorisk, men bare den tilhørende opgaveregning.
Den faste dagsorden for fredagsmøderne sikrede, at gruppen kom igennem alle aftalte punkter på hvert møde. Dette gav god mødestruktur og sikrede, at gruppen ikke sprang emner over. Derudover var fredagsmøderne med til at give overblik over status for projektet, da vi her gennemgik den overordnede tidsplan. Hermed havde gruppen et fælles overblik over projektet, hvilket er en god forudsætning for et godt samarbejde. \\
Konflikthåndteringen på fredagsmødet var et godt tiltag for gruppen. Det medførte at vi fik snakket om problemer og misforståelser, hvilket styrkede sammenholdet i gruppen. Gruppen var enige om, at konflikthåndteringen skulle foregå konstruktivt, hvorved der ikke foregik mudderkastning. Da konflikthåndtering var et punkt på dagsordenen, blev der taget problematikker op til diskussion, som gruppemedlemmer ellers potentielt havde fortiet. Dette medførte et tættere samarbejde, da der ikke var små irritationer mod hinanden.

\subsection{Grupperoller}
Gruppens valg om at have få roller i projektarbejdet skabte overblik, da alle i gruppen således var på lige fod. Ligeledes blev ingen defineret ud fra en titel, hvilket gjorde samarbejdet mere flydende og dynamisk. Da referentrollen ikke blev roteret, var kvaliteten af referaterne konsistent. Størstedelen af gruppen fik dog ikke træning i at tage referat. \\
Ordstyrerrollen blev roteret, hvormed alle fik mulighed for at prøve denne rolle til vejledermøder. Det var imidlertid ikke alle, der opfyldte denne rolle optimalt, hvilket gjorde nogle vejledermøder ustrukturerede. 

\subsection{Kommunikation}
Dokumentet med individuelle ambitionsniveauer var en god måde at blive bekendt med hinandens mål i begyndelsen af projektet. Det blev dog ikke anvendt til andet i løbet af projektet og mistede dermed sin funktion, da vi ikke har taget højde for hinandens forskelligheder nedskrevet i dokumentet. \\
Det var vores intention at afholde sociale arrangementer løbende, men grundet følt tidspres blev disse ofte tilsidesat til fordel for projektarbejde eller personlige planer. Vi fik imidlertid afholdt et par arrangementer, der gavnede sammenholdet og derved samarbejdet i gruppen. \\
Konflikthåndteringen fra fredagsmødets faste dagsorden viste sig at være et vigtigt punkt. I starten blev det ikke benyttet, men senere i projektet blev der drøftet problematikker, som ellers var blevet fortiet. Derved fik vi snakket om vores individuelle, og til tider personlige, problemer med hinanden indbyrdes i gruppen.