\chapter{Læreprocesser}
I det kommende afsnit vil læreprocessen gennem projektet og de relevante kurser blive behandlet på det beskrivende, vurderende og det reflekterende plan. 

\section{Beskrivelse}
Vi afholdte et indledende møde, da vi startede i gruppen. Her præsenterede vi os selv ved, at vi på skift snakkede om, hvilke lærings metoder, der virkede bedst for hvert enkelte gruppemedlem.

\subsection{Læringsmetode}
Vi havde en forventning afstemning, hvor vi beskrev vores individuelle ambitionsniveau. Disse fire overordnede emner skulle beskrives:
\begin{itemize}
	\item Mål for projektet (karakter).
	\item Mål for forelæsninger (Møder du op? Tidsfordeling imellem projekt og forelæsninger.)
	\item Hvilken arbejdsform der passer én bedst (alene, hjemme, stille omkring sig? kl: $8$-$16$?)
	\item Forventninger til hjemmearbejde (hvor meget, deadline, weekend?)
\end{itemize}
Da vi havde snakket om ambitionsniveauerne, kunne vi danne en samarbejdsaftale, se bilag \ref{Bilag1}, og "Tips og tricks" dokument, se bilag \ref{Bilag2}.

\subsection{Læreproces i forhold til kurser}
Da vi har forskellige erfarings niveauer i gruppen har der været delte holdninger til kursernes relevans. Af helbredsessige årsager har enkelte medlemmer været forhindret i at deltage til forelæsninger. Der blev derfor aftalt at, opgaveregning skulle være obligatorisk for alle gruppens medlemmer, men at forelæsninger valgfrit. Det kræves dog, at medlemmerne møder forberedt op til opgaveregning. \\
Vi benyttede os meget af tavleregning i starten af perioden med calculus, hvor vi skiftes til at gå til tavlen. I ABIO og FVM-kurset blev opgaverne gennemgået på gruppen, hvorefter resultatet blev tjekket, og i tilfælde af fejl blev opgaven diskuteret. I nogle tilfælde var det dog fordelagtigt at gå i mindre undergrupper. Ved FTP-kursen var der praktiske øvelser, hvor gruppen var delt op i to, efterfulgt af en fremlæggelse af opgaverne. Vi nåede dog ikke alle opgaverne under selve kurset. 
  
\subsection{Læreproces i forhold til projektarbejde}
Igennem projektperioden har der ikke været optimal fokus på hinandens læreprocesser. Desuden har grupperumarbejde været udfordrende på nogle punkter, da gruppens medlemmer arbejder bedst på forskellige måder. Gruppen har være arbejdet med i roterende undergrupper.
%Foruden at sikre et ens sprog og en godt overblik sikre dette også en fornuftig videndeling gruppemedlemmerne imellem. 
\clearpage
\section{Vurdering}
Her opstilles de beskrevne emner i positive og negative erfaringer.
\begin{table}[h]
	%\caption{This Is a Table\label{LABEL}}
	\begin{tabular}{|p{3cm}|p{5cm}|p{5cm}|}
		\hline
		&\textbf{Positive erfaringer}&\textbf{Negative erfaringer}\\ \hline
	Læringsmetode	& Grundig forventningsafstemning. \newline Samarbejdsaftale samt tips og ticks.& Forventningsafstemning blev ikke taget lige seriøst af alle gruppemedlemmerne. \newline Forventningsafstemningen: ikke udnyttet fulde potentiale.\newline
	 Regler/guidelines fra tips og tricks blev ikke altid fulgt. \\ \hline
	Læreproces i forhold til kurser & Obligatorisk opgaveregning. \newline  Opgaver regnet på tavlen. & FTP-kurset nedprioteret i forhold til projekt under statusseminar. \newline Praktiske FTP-øvelser kunne godt virke forhastet. \\ \hline
	Læreproces i forhold til projektarbejde & Fordeling af projektarbejde i undergrupper. \newline Review på arbejdsopgaver & Optimale læreprocesser/metoder blev ikke udnyttet i tilstrækkeligt omfang. \newline Retning kunne blive ensidet ens. \\ \hline
		
	\end{tabular}
\end{table}

\section{Reflektion}
Der bliver i det følgende afsnit reflekteret over emnerne beskrevet i den beskrivende del.

\subsection{Læringsmetode}
Det var godt at få afstemt vores forventninger til hinanden, da alle kom med forskellige baggrund og erfaringer. Dette gjorde, at vi fik et indblik i, hvordan de enkelte af os lærte bedst. Dog blev der ikke taget hensyn til dette under hele semestret, da tiden spidsede til. Forventningsafstemningsopgaven gav en god diskussion omkring vores individuelle arbejdsmetoder, og eventuelte emner, som gruppens medlemmer gerne ville have vendt.  

\subsection{Læreproces i forhold til kurser}
Det fungerede godt at vi arbejdede med opgaverne sammen på tavlen i calculus, da metoden var optimal for alle og gav en god forståelse af opgaverne. Herved fik alle desuden muligheden for at regne opgaver med hjælp fra gruppen. Jo længere vi kom i semestret, fyldte tavleregningsdelen dog mindre, da nogle medlemmer regnede hurtigere end andre. Dette havde både en negativ og positiv effekt, da de hurtige kom længere i opgaverne og kunne hjælpe de resterende. Dog kunne resten føle sig bagud og fortabt, fordi vi ikke længere snakkede om opgaverne på tavlen. I FTP-kurserne var det et problem, at alle opgaverne ikke blev taget lige seriøst grundet tidspres fra projektet. Der var delte meninger om, hvordan der skulle tages hånd om problemet. Dette resulterede i, at opgaverne enten måtte laves individuelt eller udskydes til eksamensforberedelse.
 
\subsection{Læreproces i forhold til projektarbejde}
Den manglende fokus på det enkelte gruppemelslems fortrukne arbejds-/læringsstil har til tider resulteret i et mindre produktivt arbejdsmiljø. Det var en udfordring for gruppen at finde en balance igennem hele forløbet. Dette har resulteret i, at vi skulle indgå kompromis med hinanden, hvorved der har været meget hjemmearbejde istedet for gruppearbejde. \\
Opdelingen af gruppen og roterring i undergrupper til retteprocessen har gavnet produktiviteten, da det sikrede god vidensdeling og en ny vinkel på opgaven. Man kunne dog miste den røde tråd igennem afsnittene, da der var forskellig forståelse og forfattere til afsnittet. Dette skabte en større retteproces i sidste ende, hvilket kunne være undgået, hvis man havde rettet hele afsnit af gangen. \\
I FTP-kursernes praktiske øvelser fik vi erfaring med udstyret, som vi skulle benyttet til vores projekt forsøg. Dette var en meget god oplevelse for gruppen, da man derved havde erfaring til at udføre sit projektforsøg.

\section{Synthese}
I P$3$ vil det være en fordel at være mere opmærksom på arbejdsprocessen, og gruppens medlemmer skal være forberedt på et kompromis. Vi vil fortsat arbejde i roterende undergrupper og arbejde sammen til opgaverne i kurserne.