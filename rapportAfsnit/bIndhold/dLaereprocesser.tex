\chapter{Læreprocesser}
I det kommenende afsnit vil læreprocessen gennem projektet og de relevante kurser blive behandlet på det beskrivende, vurderende og det reflekterende plan. 
\section{Beskrivelse}
Da vi startede i gruppen, sattes vi os og holdte et indledende møde hvor vi præsenterede os selv, herefter to vi en runde hvor vi snakkede om hvilke lærings metoder der virker bedst for hvert enkelte af os.
\subsection{Læringsmetode}
 Vi udarbejdede en \fxnote{refference til bilag med vores forventnings afstemning} forventning afstemning, et dokument hvor vi satte ord på vores ambitionsniveau. Der var fire over ordnede emner vi skulle dække:

\begin{itemize}
	\item Mål for projektet (karakter)
	\item Mål for forelæsninger (Møder du op? Tidsfordeling imellem projekt og forelæsninger)
	\item Hvilken arbejdsform der passer én bedst (alene, hjemme, stille omkring sig? kl: 8-16?)
	\item Forventninger til hjemmearbejde (hvor meget, deadline, weekend?)
\end{itemize}
Da vi haverisær havde udarbejdet opgaven, gennem gik vi punkterne på et fredagsmøde som var med til at danne grundlaget for vores samarbejdsaftale. \fxnote{reff til samarbejdsaftale og tips og ticks} 

\subsection{Læreproces i forhold til kurser}
Da vi i gruppen har forskellige erfarings niveauer, har ikke alle kurser været lige essentielle for alle. Af helbredsessige årsager har enkelte medlemmer været forhindret i at deltage i forelæsninger. Der blev derfor aftalt at opgaveregning skulle være obligatorisk for alle gruppens medlemmer, men at forelæsninger valgfrit. Det kræves dog at medlemmerne møder forberedt op til opgaveregning. I starten af Calcslus benyttede vi meget tavlen til at regne opgaverne, her skiftes vi til at regne. Hvorved alle fik muligheden for at regne opgaver, med hjælp fra gruppen. Samme process var gældende for ABIO-kurset hvor opgaverne ligeledes blev taget side for side og diskuteret hvorefter resultatet blev sammenholdt med rettelisten, hvis der var uligheder blev disse gennemgået og fejlen/misforståelsen korrigeret.
FVM-kurset blev behandlet på samme måde, opgaverne er blevet løst på gruppen på en hertil afsat dag. Her var det ofte fordelagtigt at inddele studiegruppe i mindre undergrupper, for at arbejde på delene af opgaven for så at gennemgå delemnerne på gruppen og færdiggøre opgaven. Ved FTP-kursen var der foruden forelæsninger også praktiske øvelser efterfulgt af en fremlæggelse af opgaverne. Ved de praktiske øvelser var gruppen ofte delt i 2 for at alle medlemmerne havde mulighed for at kunne få erfaring med udstyret der var tilrådighed for os. Vi nåede på gruppen ikke alle opgaverne under selve kurset, da der fra projektets side var et stort pres i forbindelse med statusseminar.  
\subsection{Læreproces i forhold til projektarbejde}
Der har igennem projektperioden ikke været optimal fokus på hinandens læreprocesser, grupperumarbejde har på nogle punkter været udfordrende da arbejdsstilene har været forskellige. Hvilket har resulterede i at nogle har arbejdet hjemme, imens andre har arbejdet i grupperummet da distraktion har været et problem. Processanalysedelen har være arbejdet med i roterende undergrupper, nogle gange med enkelte emner til enkelte personer. Men ligeledes i undergrupper af to og tre personer, rette processen gennemgående roteret medlemmerne i mellem således at et afsnit ikke er blevet rettet af en enkelt person eller af de samme personer som har skrevet det pågældende afsnit. 
%Foruden at sikre et ens sprog og en godt overblik sikre dette også en fornuftig videndeling gruppemedlemmerne imellem. 
\section{Vurdering}
\begin{table}[h]
	\begin{tabular}{|l|l|l|}
		\hline
		& Positive erfaringer & Negative erfaringer \\ \hline
		Læringsmetode & \begin{tabular}[c]{@{}l@{}}
			\begin\{itemize\}\\ 
			\item Grundig forventnings afstemning.
			\item Fastsat regler samt tips og ticks.\\ \end{itemize\}\end{tabular} & \begin{tabular}[c]{@{}l@{}}\begin{itemize\}\\ 
			\item Opgaven blev ikke taget lige seriøst af alle gruppemedlemmerne, og den blev ikke benyttet til dens fulde potentiale.
			\item Regler/guidelines fra tips og tricks blev ikke altid fulgt, en håndhævelse her ville måske være en idé. \\ \end{itemize\}\end{tabular} \\ \hline
		Læreproces i forhold til kurser & \begin{tabular}[c]{@{}l@{}}
			\begin{itemize\}\\ 
				\item Obligatorisk opgaveregning
				\item Opgaver regnet på tavlen \\ \end\{itemize\}\end{tabular} & \begin{tabular}[c]{@{}l@{}}
				\begin{itemize\}\\ 
				\item Kurset blev nedprioteret i forhold til projekt under statusseminar.\\ 
				\item Praktiske øvelser kunne godt virke forhastet. \\ \end\{itemize\}
				\end{tabular} \\ \hline
		\multicolumn{1}{|c|}{Læreproces i forhold til projektarbejde} & \begin{tabular}[c]{@{}l@{}}
			\begin{itemize\}\\ 
				\item Fordeling af projektarbejde i undergrupper.\\ \item Opsamling på arbejde.\\ 
				\item Review på arbejdsopgaver\\ \end{itemize\}\end{tabular} & \begin{tabular}[c]{@{}l@{}}
				\begin{itemize\}\\ 
				\item Optimale læreprocesser/metoder blev ikke udnyttet i tilstrækkeligt omfang. 
				\item Retteprocess kunne blive ensidet hvis rotationen ikke blev fulgt. \\ 
				\end{itemize\}\end{tabular} \\ \hline
	\end{tabular}
\end{table}
\section{Reflektion}
\subsection{Læringsmetode}
Da vi alle kom med forskellige baggrund og erfaringer, var det godt at får afstemt vores forventninger til hinanden. Dette gjorde at vi hurtigere fik et indblik i hvordan de enkelte af os lærte bedst, det var dog ikke under hele semesteret at dette var i fokus. Selvom opgaven kunne virke omsonst giver den stadig struktur for en god diskussion omkring vores individuelle arbejdsmetoder, og eventuelte emner som gruppemedlemmerne gerne ville have vendt.  
\subsection{Læreproces i forhold til kurser}
Det fungerede godt at vi arbejdede med opgaverne som en samlet gruppe, metoden med at regne opgaver for hinanden var langsom men gav en rigtig god forståelse af opgaverne, som semesteret skred frem fyldte tavleregningsdelen dog mindre og mindre, det havde både en negativ og positiv effekt. Efter som vi blev bedre til emnerne blev det mind og mindre nødvendigt, men det skyldes til dels også dovenskab. I FTP-kurserne var problemet at alle opgaverne ikke blev taget lige seriøst, dette resulterede at opgaverne enten måtte laves individuelt eller måtte udskydes til eksamensforberedelse. Der var delte meninger om hvordan problemet skulle takles, grundet stress og tidsmangel blev opgaverne ikke lavet på gruppen. 
\subsection{Læreproces i forhold til projektarbejde}
Den manglende fokus på de enkelte gruppemelslemmernes fortrukne arbejds-/læringsstile har tiltider resulteret i et mindre produktivt arbejdsmiljø. Det har været en udfordring for gruppe at finde en balance igennem hele forløbet og det har været et spørgsmål om kompromis, nogle dage har været en blanding af gruppemøde og gruppe arbejde mens andre dage har været del gruppemøde og hjemmearbejde. Strukturen med opdeling af gruppen i undergrupper har gavnet produktiviteten, og rottering af rattearbejdet har sikret god vidensdeling gruppen imellem. Hvis den rettende undergruppe har haft problemer med at forstå et afsnit er afsnittet blevet forklaret af forfatteren, så der ikke er sket misforståelser eller fejretning. Ulempen ved at være flere undergrupper er at den rådetråd nemt kan mistes og sproget rapporten har tildens til at blive forskelligt. Hvilket skaber er større samlede rattearbejde tilsidst i projektet for at rette op på dette. 
\section{Synthese}
Det ville være en fordel at være mere opmærksom på arbejdsprocessen, for at optimerer arbejdet, selvom det tiltider må resulterer i kompromis. Uddeling af arbejdet med efterfølgende review og gennemretning har vist sig at været en fornuftig metode og vil også benyttes fremover. En bedre fordeling af tiden, således så at kursusopgaverne får en højere prioritering vil være noget vi skal have fokus på fremover.