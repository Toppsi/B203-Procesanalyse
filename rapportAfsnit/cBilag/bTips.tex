  \chapter{Bilag - Tips og Tricks}
  \label{Bilag2}
    

\begin{itemize}

      \item Fredagsmøde:
	
 	
 	
 	\begin{itemize}
 	    \item Evaluering af nye arbejdsformer. 
 	    \item Hver anden uge evalueres samarbejdsaftalen og tips \& tricks
 	    \item Arbejdsopgavens formulering stiller kravet for opgavens niveau. Man arbejder efter bedste evne.
 	    \item Hvis uenighed ikke kan løses hurtigt ved kompromis stemmes der på gruppen.
 	 \begin{itemize}
        \item Skulle problemerne eskalere til personlige problemer løses det hos de implicerede parter. Skulle disse parter ikke løse   konflikten selv tages det op med vejleder.
        Kan konflikten ikke løses er sidste udvej ekskludering af gruppen
    
    \end{itemize}  
       \item Startmøde skal indeholde:
    \begin{itemize}
    	\item Hurtig gennemgang af ”næste møde” fra forrige gang
    	\item Hvis hjemmearbejde gennemgang af dette 
    	\item Gennemgang af hvad det enkelte gruppemedlem laver
    	
    \end{itemize}   
 	    \item Slutmøde skal indeholde:
 	   
 	\begin{itemize}  
 	   \item Gennemgang af dagens arbejde 
 	   \item  Diskussion af hvad der dagsordenen skal være til næste møde
 	   \item Eventuelt uddeling af hjemmearbejde   
 	
   \end{itemize}
   
     
   	 \item Vi prøver en arbejdsgang der heder 20 min arbejde fulgt af 5 min pause. vi forsøger dette i en 2 ugers periode hvorefter vi vurder om det er godt
   	 \item Vi laver versioner af afsnit som sendes til review hos en anden del af gruppen og rettelserne laves
   	 \item Hvis folk ikke møder til aftalt tid og ikke har givet besked ringes de op kl. 9.45
   	 \item Dead lines for hjemmeopgaver, laves selv men godkendes af gruppen. Både tid og mængde. 
   	 \item Hvad skal vi gøre, så vi alle ikke skal læse alle kilder i forhold til referat af dem? Man markerer det vigtigste med overstregning i PDF’en, ligger den ind i dropbox og skriver referat samt diverse vigtige informationer inde på google drev.
   	 
   \end{itemize}
   
  \end{itemize}
  
  \clearpage