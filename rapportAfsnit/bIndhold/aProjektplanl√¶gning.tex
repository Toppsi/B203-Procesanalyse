\chapter{Projektplanlægning}

\section{Beskrivelse}

\subsection{Valg af projekt}
I forbindelse med valg af projekt udarbejdede vi hver især en prioriteringsliste ud fra projektkataloget. Derefter fandt vi i fællesskab frem til et kompromis. Projektvalget faldt på flertallets 2. prioritet.

\subsection{Tidsplanlægning}
I P2 havde vi fire forskellige projektplaner; En overordnet tidsplan for hele projektperioden, en tavle med en tidsplan for de følgende to uger, en dagsorden for den pågældende dag samt en to-do liste der nogle gange afløste dagsordnen. 
Det fælles formål for projektplanerne var at give et overblik over projektet på kort og på lang sigt.
Den overordnede tidsplan skulle give et overblik over hele projektperioden, imens tidsplanen på tavlen blev udarbejdet ud fra den overordnede plan for mere specifikt at beskrive hvad der skulle laves i løbet af de følgende to uger. Dagsordenen havde til formål at udspecificere hvad programmet for den enkelte dag var. Sidst i projektperioden blev dagsordenen dog ofte erstattet af en to-do liste med de opgaver der manglede at blive udført for at projektet kunne blive færdiggjort.

\subsection{Projektledelse}
Vi havde i projektperioden ikke én decideret projektleder, men anvendte i stedet hinandens kompetencer når vi sad med arbejdsopgaver vi ikke selv kunne løse.

\subsection{Eksperimenter}
I forbindelse med arbejde i grupperummet forsøgte vi at arbejde ud fra 20-5 metoden, hvor vi har arbejdet koncentreret i 20 minutter og efterfølgende holdt fem minutters pause. Der var sat alarm på de 20 minutter men ikke på de fem minutters pause. 

\subsection{Rapportopsætning}
Under udarbejdelsen af projektet anvendte vi GitHub til at dele vores arbejde på. Projektet blev udarbejdet i LaTeX. 

\section{Vurdering}

\subsection{Valg af projekt}

\subsubsection{Positive erfaringer}
Vi nåede frem til et kompromis som alle var tilfredse med i forhold til hvilket projekt vi skulle arbejde med.
\subsubsection{Negative erfaringer}

\subsection{Tidsplanlægning}

\subsubsection{Positive erfaringer}
Den overordnede tidsplan fungerede godt i begyndelsen og hjalp os med at komme hurtigt igang. Tidsplanen på tavlen fungerede godt og gav et godt overblik over hvad der skulle laves i de pågældende uger. Dagsordnen gav et godt overblik over hvad der skulle laves den pågældende dag således at alle var enige herom. Vi arbejdede effektivt ud fra to-do listen. 

\subsubsection{Negative erfaringer}
Den overordnede tidsplan mistede sin funktion efter statusseminaret. Vi endte med at have lavet tre forskellige to-do lister hvilket var uoverskueligt.

\subsection{Projektledelse}

\subsubsection{Positive erfaringer}
I stedet for at have én person der stod med alt ansvaret udnyttede vi hinandens kompetencer indenfor forskellige områder.

\subsubsection{Negative erfaringer}
Ved nogle af de svære arbejdsopgaver var det svært at finde ud af hvem man skulle søge hjælp hos.

\subsection{Eksperimenter}

\subsubsection{Positive erfaringer}
Det fungerede godt at arbejde koncentreret i 20 minutter - vi var effektive og fik udført mange arbejdsopgaver når vi anvendte metoden.

\subsubsection{Negative erfaringer}
Planen om at holde fem minutters pause blev ofte ikke overholdt - pauserne blev tit længere.

\subsection{Rapportopsætning}

\subsubsection{Positive erfaringer}
Ved brugen af GitHub var alle gruppemedlemmer altid opdateret på hvad hinanden arbejdede med hvilket skabte et godt overblik. Det fungerede desuden godt at skrive projektet i LaTeX.

\subsubsection{Negative erfaringer}
I begyndelsen af projektperioden var det vanskeligt at anvende GitHub og LaTeX for nogle af gruppens medlemmer. 

\section{Refleksion}

\subsection{Valg af Projekt}
Vi nåede frem til et tilfredsstillende kompromis i forhold til valg af projekt. Vi havde alle skrevet en prioriteringsliste og var på den måde klar til at komme med argumenter for hvorfor det ville være mest interessant at arbejde med de projekter vi hver især havde valgt. På den måde fik vi allerede første dag i projektperioden en god diskussion og kom omkring stort set alle projekterne i kataloget inden vi kunne træffe den endelige beslutning. 

\subsection{Tidsplanlægning}
Den overordnede tidsplan fungerede godt i begyndelsen da oversigten gav os et overblik over hvad vi skulle gå igang med allerede fra første dag. Efterfølgende fungerede den også fint, men da vi nåede til statusseminaret og blev bevidste om at en stor del af projektet burde omstruktureres holdt den overordnede plan ikke længere, og mistede dermed sin funktion i den resterende del af projektet. 
Tidsplanen på tavlen gav et godt overblik over projektarbejdet da det overordnet var beskrevet hvad vi skulle lave de enkelte dage. Desuden var kurserne skrevet ind i tidsplanen, hvilket gjorde det lettere for os at overskue hvor meget tid vi overordnet set havde til projektarbejde i de pågældende uger. 
Dagsordenen blev udarbejdet på baggrund af tidsplanen på tavlen, og var på den måde medvirkende til at udspecificere hvad der skulle laves den pågældende dag for at vi nåede det vi skulle. Dette gav et godt overblik, og skabte struktur på de enkelte arbejdsdage.
To-do listerne var gode til at give et overblik over hvilke arbejdsopgaver der manglede for at projektet kunne blive færdiggjort, og var medvirkende til at vi arbejdede effektivt. Det fungerede imidlertid ikke godt at vi endte ud med at have tre forskellige to-do lister, da vi på den måde af og til mistede overblikket over hvad vi egentlig manglede og om arbejdsopgaverne på den foregående to-do liste allerede var færdiggjort.

\subsection{Projektledelse}
Det har i vores gruppe fungeret godt ikke at have en projektleder, da vi overordnet set hver især har haft nogle gode kompetencer inden for forskellige områder. Derfor har der ikke været én person i gruppen der har kunne hjælpe med alle arbejdsopgaverne, men man har i stedet haft muligheden for at vurdere hvem i gruppen det har været mest optimalt at søge hjælp hos, ud fra den opgave man har siddet med. 
En ulempe har været, at der ved nogle af de svære arbejdsopgaver ikke har været én person i gruppen der har haft en særlig viden indenfor området. I disse situationer var det vanskeligt at vurdere hvem man skulle søge hjælp hos, da der ikke har været én leder som man altid kunne gå direkte til. I langt de fleste tilfælde har disse problemer dog kunne løses på et gruppemøde, hvor hele gruppen har hjulpet med at nå frem til en måde hvorpå arbejdsopgaven kunne løses.

\subsection{Eksperimenter}
Det fungerede godt at arbejde koncentreret i 20 minutter ad gangen, da dette var et overskueligt tidsrum at holde fuldt fokus i. På den måde undgik man at blive ukoncentreret og komme til at forstyrre de øvrige gruppemedlemmer i deres arbejde. Det fastsatte tidsrum var også medvirkende til at man kunne sætte sig personlige mål for hvor meget man skulle nå at have lavet før der igen var pause, så på den måde blev arbejdet effektivt. 
Pausen der burde vare fem minutter blev ofte længere, hvilket formentlig skyldtes at der ikke blev sat en alarm til at påminde os om at påbegynde arbejdet igen. Dermed opstod en tendens til at vi glemte tiden i pauserne, hvilket ikke var optimalt, da vi kunne have brugt den ekstra pausetid på at arbejde i stedet.

\subsection{Rapportopsætning}
Anvendelsen af GitHub gav et godt overblik over hvad de enkelte gruppemedlemmer arbejdede på, da man hver gang man gemte sit arbejde skulle tilføje det til fællesmappen og på GitHub lave en kort beskrivelse af hvad man havde lavet. 
At det i begyndelsen af projektperioden var vanskeligt for nogle gruppemedlemmer at anvende GitHub og LaTeX skyldtes formentlig at det var programmer de ikke havde anvendt før. Dette gav nogle problemer på GitHub, da filerne nogen gange ikke blev tilføjet korrekt til fællesmappen. Dette var dog ikke længere et problem i samme omfang da vi kom længere ind i projektperioden og alle fik bedre styr på programmerne. 