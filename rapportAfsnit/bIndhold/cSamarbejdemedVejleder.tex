\chapter{Samarbejde med Vejleder}
I det følgende afsnit vil samarbejdet med vejleder blive beskrevet, vurderet og analyseret. 

\section{Beskrivelse}
I dette afsnit beskrives samarbejdsaftalen, som blev indgået med vejleder. Derudover vil dagsordenen samt typen af vejledning blive gennemgået.

\subsection{Samarbejdsaftale}
I starten af P$2$ indgik vi en samarbejdsaftale med vejleder, se bilag \ref{Bilag3}. Denne samarbejdsaftale var skriftlig og bestod af retningslinjer for afsendelse af dagsorden, referat af forrige vejledermøde samt materialer, som vi forventede at få feedback på. Vi aftalte derudover, at vi havde valgfrit vejledermøde hver onsdag.

\subsection{Dagsorden}
Vi udarbejdede en dagsorden til hvert vejledermøde. Dagsordenen indeholdte punkter som f.eks. update på projekt og spørgsmål, der krævede vejledning. Den blev lavet fælles på gruppen, hvor hver gruppemedlem kunne komme med spørgsmål. Der blev skrevet referat under vejledermødet af referenten, som blev sendt til vejleder efter mødet.

\subsection{Mundtlig og skriftligvejledning}
Under P$2$ har vi eksperimenteret med både mundtlig og skriftlig vejledning. Under mødet optrådte et gruppemedlem som ordstyrer, der startede mødet og uddelegerede spørgsmålene iblandt de andre medlemmer, som var tilsendt i dagsordenen. Den skriftlige vejledning omhandler vejleredens feedback på materialen, som blev sendt til hende på mail.

\section{Vurdering}
\begin{table}[h]
	\begin{tabular}{|l|l|l|}
		\hline
		\textbf{}                                & \textbf{Positive erfaringer}                                                                      & \textbf{Negative erfaringer}                                                                                   \\ \hline
		\textbf{Samarbejdsaftale}                & \begin{tabular}[c]{@{}l@{}}Samarbejdsaftale med vejleder\\ Valgfrit møde hver onsdag\end{tabular} & Samarbejdsaftale med vejleder                                                                                  \\ \hline
		\textbf{Dagsorden}                       & \begin{tabular}[c]{@{}l@{}}Update i mail\\ Referat\\ Mødets gang\end{tabular}                     & \begin{tabular}[c]{@{}l@{}}Referatet blev ikke udnyttet optimalt\\ Lange spørgsmål ud fra et emne\end{tabular} \\ \hline
		\textbf{Mundtlig og skriftligvejledning} & \begin{tabular}[c]{@{}l@{}}Møde selvom sygdom via skype\\ Type af vejledning\end{tabular}         & Sprogbarriere                                                                                                  \\ \hline
	\end{tabular}
\end{table}

\section{Reflekion}
\subsection{Samarbejdsaftale}
Det var godt at have en samarbejdsaftale med vejleder, da dette gjorde, at der var enighed om det kommende samarbejde. Derudover er det fordelagtigt at nedskrive denne aftale i stedet for en mundtlig forventningsafstemning, da mulige uenigheder derved kan undgås. Samarbejdsaftalen blev dog ikke altid overholdt af vejleder, da e-mails ikke altid blev besvaret, samt referat fra forrige vejledermøde ikke blev gennemlæst. Vi skulle dog have gjort vejleder opmærksom på dette. \\
Vejleder sagde fra starten, at hun havde tid til et møde hver onsdag. Dette var valgfrit, hvilket var godt, da gruppen derved selv skulle tage stilling til, om vi havde brug for vejledning.

\subsection{Dagsorden}
Vi havde gode erfaringer med ..
Dagsordenen gjorde vejlederen og os forberedt til mødet
Referaterne kunne blandt andet bruges til at informere de fraværende gruppemedlemmer om hvad der var blevet snakkes om.


\subsection{Mundtlig og skriftligvejledning}