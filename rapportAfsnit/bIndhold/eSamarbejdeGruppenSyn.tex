\section{Syntese}
Der er stadig delte meninger om opdelingen af samarbejdsaftalen, da de forskellige gruppemedlemmer vurderer de positive og negative sider ved opdelingen forskelligt. Nogle gruppemedlemmer foretrækker derved en mere samlet samarbejdsaftale i P3, hvor andre ønsker en mere opdelt og kortfattet samarbejdsaftale.
Det er i P3 vigtigt at tage samarbejsaftalen op til vurdering flere gange i projektperioden. På denne måde opdateres aftalen og alle gruppemedlemmer bliver påmindet om retningslinjerne. 
At afholde et ugentligt møde, der skaber overblik samt tager hånd om konflikter vil fortsat blive benyttet på P3. Strukturen på fredagsmødet kan imidlertid med fordel ændres således at der ved nogle punkter tages ved en runde frem for fri snak. Dette sikrer at alle kommer til orde i svære diskussioner. 
Vi har været glade for vores fordeling af roller og vil i P3 arbejde videre på dette system. Referantrollen bør evt. roteres, og ordstyreren bør hjælpes med retningslinjer for at strømline møderne.
Der er i gruppen enighed om at der i P3 bør arbejdes mere med de individuelle forventninger til projektet og hinanden, da der på måde tages hensyn til det enkelte gruppemedlems behov. Vi vil forsøge at prioritere sociale arrangementer højere end i denne projektperiode, da vi erfarede at disse havde en positiv effekt på gruppesamarbejdet.  