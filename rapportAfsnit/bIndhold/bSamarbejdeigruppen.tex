\chapter{Samarbejde i Gruppen}
For at sikre et godt samarbejde i gruppen er det nødvendigt at blive enige om, hvordan arbejdet skal forløbe. Dette kræver at vi forventningsafstemmer, danner en samarbejdsaftale og er opmærksomme på at have god kommunikation. Disse vil i de følgende afsnit blive beskrevet, reflekteret og analyseret.

\section{Beskrivelse}
I dette afsnit vil de enkelte faktorer for gruppens samarbejde med hinanden blive beskrevet. 

\subsection{Samarbejdsaftale}
I starten af projektperioden udarbejdede gruppen en samarbejdsaftale. Der var uenighed om, hvad en samarbejdsaftale skulle indeholde og hvordan den skulle sættes op. Derfor lavede gruppen både en samarbejdsaftale og et dokument kaldet "Tips og Tricks". Disse er vedlagt i bilag \ref{Bilag1} og bilag \ref{Bilag2}. Samarbejdsaftalen var krav og forventninger til gruppens medlemmer. Dokumentet "Tips og Tricks" var retningslinjer til arbejdsformer og metoder. Samarbejdsaftalen var fast og blev ikke ændret igennem hele projekforløbet, hvorimod "Tips og Tricks" blev tilpasset i takt med at gruppen fandt frem til mere optimale arbejdsmetoder. 

\subsection{Gruppemøder og Gruppearbejde}
Gruppen vedtog i starten af projektperioden, at vi skulle mødes hver dag fra kl. $9$.$15$ til $16$.$15$, medmindre andet var aftalt. Sluttiden varierede i begyndelsen af projektperioden og blev sat individuelt. Senere i projektet blev mødetiden ændret til $8.15$ og den fastsatte sluttid blev overholdt. Der var i starten ingen aftale om fremmøde til semestrets tilhørende kurser, men det blev senere vedtaget, at det var obligatorisk for gruppens medlemmer at møde op til kursernes opgaveregning.
Gruppen har derudover gjort brug af tre forskellige møder; fredagsmøder, vejledermøder og statusmøder. \\
Fredagsmøde havde til formål at opsummere ugens arbejde, overholdelse af tidsplan og konflikthåndtering. Til disse møder var der nedskrevet en fast skabelon for dagsorden, der kan ses i bilag \ref{Bilag4}. \\
Vejledermøde var et møde mellem gruppen og vejleder, der vil blive gennemgået i det kommende kapitel. \\
Statusmøde var et kort møde med en fast dagsorden, der blev afholdt i starten og slutningen af hver dag. Dagsordenen for dette møde indbefattede en runde, hvor hvert gruppemedlem gennemgik status for eget arbejde.

\subsection{Grupperoller}
Gruppen gjorde brug af to roller ved møder og gruppearbejde; ordstyrer og referent. Ordstyrer styrede samtalen til møder, stoppede ligegyldige eller langtrukne diskussioner og kaldte til fælles pause. Referenten tog referat ved møder og skrev dagsorden for møderne. Gruppen forsøgte at undgå for mange forskellige roller.

\subsection{Kommunikation}
Gruppen startede ud med at snakke om vores individuelle ambitionsniveau og forventninger for dette projekt. Der blev udarbejdet nogle punkter, som hvert gruppemedlem skulle komme ind på. Alt der blev nævnt, blev skrevet ned i et fælles dokument, som alle gruppemedlemmer havde adgang til. Derudover blev gruppen enige om at prioritere sociale arrangementer med hinanden, og hver fredag havde gruppen et punkt på dagsordenen, der indbefattede konflikthåndtering. 

\section{Vurdering}
For at give et bedre overblik over gruppens erfaringer med anvendte metoder, opstilles disse i \tableref{Samvur}.

\begin{table}[h]
	\caption{}
	\label{Samvur}
	\begin{tabular}{|p{3.5cm}|p{5cm}|p{5cm}|}
		
		\hline
		&\textbf{Positive erfaringer}                               &  \textbf{Negative erfaringer}                                                               \\ \hline
		\textbf{Samarbejdsaftale}          
		&  
		Samarbejdsaftalen var kort og præcis.                &  Opdelt samarbejdstale var uoverskuelig.                  \\ \hline
		\textbf{Gruppemøder og Gruppearbejde} &  
		Den faste mødetid gav overblik over samarbejdet.                                                                                                       \newline
		Obligatorisk opgaveregning gav fælles ansvar for indlæring.                                            \newline
		Fast dagsorden gav struktur på fredagsmøderne.                                                                                           \newline
		Fredagsmødet gav gruppen overblik.                                                                                        \newline
		Fredagsmødet muliggjorde konstruktiv personlig kritik.
		&   Det var svært at arbejde koncentreret fra kl. $8.15$-$16.15$.                                              \\ \hline
		\textbf{Grupperoller}                         &
		Få roller gav overblik over gruppestrukturen. \newline
		Referentrollen blev ikke roteret.                                                                                                                       \newline
		Ordstyrerrollen blev roteret.                                                                                                  \newline
		&   Referentrollen blev ikke roteret.                                                               \newline
		Ordstyrerrollen blev roteret.                                       \\ \hline
		\textbf{Kommunikation}                        &   
		Kendskab til individuelle ambitionsniveauer.     \newline
		Nedskrivning af forventningsafstemning.                                                              \newline
		Sociale arrangementer.                                                   \newline
		Konstruktiv konflikhåndtering.                                                  \newline
		&   Dokumentet med ambitionsniveauer blev ikke anvendt. \newline
		Der blev afholdt få sociale arrangementer.                                                       \newline
		\\ \hline
	\end{tabular}
\end{table}
\clearpage
\section{Refleksion}
I dette afsnit vil de enkelte faktorer for gruppens samarbejde med hinanden blive reflekteret.

\subsection{Samarbejdsaftale}
At have samarbejdsaftalen på skrift var positivt, da alle gruppemedlemmer derved tog aftalen alvorligt. Derudover gav det enighed om, hvilke regler gruppearbejdet skulle overholde. Dette førte til et bedre samarbejde, da alle gruppemedlemmer var enige om og indforstået med kriterierne for et godt samarbejde. \\
Opdelingen af samarbejdsaftalen var et kompromis i gruppen og havde både positive og negative konsekvenser. Selve samarbejdsaftalen blev kort og præcis, hvorved den blev lettere at overholde. Dokumentet "Tips og Tricks" blev dog taget mindre seriøst, da det blev tolket som forslag og ikke retningslinjer af nogle gruppemedlemmer. Opdelingen medførte, at samarbejdsaftalen og "Tips og Tricks" blev svære at overskue.

\subsection{Gruppemøder og Gruppearbejde}
Den ustrukturerede mødetid i starten af projektperioden havde en negativ indflydelse på de enkeltes arbejdsindsats, da flere foretog overspringshandlinger. Den senere fastsatte sluttid gav således overblik over samarbejdet, da det var muligt for alle gruppemedlemmer at følge med i hinandens arbejde. Dette gav tryghed for nogle gruppemedlemmer, da det var muligt at se projektets progressive forløb. \\ %Det er dog vanskeligt at holde fuld koncentration hele dagen, hvilket medførte et splittet og opdelt arbejdsflow.
Opgaveregningen for semestres tilhørende kurser blev gjort obligatorisk for alle gruppemedlemmer, da fremmødet hertil var mangelfuldt. Det lille fremmøde virkede demotiverende og vanskeliggjorde opgaveregningen, hvormed gruppen blev enige om at gøre den obligatorisk. Hermed tog gruppen et fælles ansvar for indlæring, hvilket, udover at forbedre forståelsen for kurserne også, gav bedre sammenhold i gruppen. \\ %Denne regel tog dog ikke hensyn til gruppemedlemmernes forskellige foretrukne arbejdsmetoder, hvilket kan have hæmmet indlæringen ved nogen. Gruppen forsøgte at tage højde for dette, da fremmøde til forelæsninger ikke blev gjort obligatorisk, men bare den tilhørende opgaveregning.
Den faste dagsorden for fredagsmøderne sikrede, at gruppen kom igennem alle aftalte punkter på hvert møde. Dette gav god mødestruktur og sikrede, at gruppen ikke sprang emner over. Derudover var fredagsmøderne med til at give overblik over status for projektet, da vi her gennemgik den overordnede tidsplan. Hermed havde gruppen et fælles overblik over projektet, hvilket er en god forudsætning for et godt samarbejde. \\
Konflikthåndteringen på fredagsmødet var et godt tiltag for gruppen. Det medførte at vi fik snakket om problemer og misforståelser, hvilket styrkede sammenholdet i gruppen. Gruppen var enige om, at konflikthåndteringen skulle foregå konstruktivt, hvorved der ikke foregik mudderkastning. Da konflikthåndtering var et punkt på dagsordenen, blev der taget problematikker op til diskussion, som gruppemedlemmer ellers potentielt havde fortiet. Dette medførte et tættere samarbejde, da der ikke var små irritationer mod hinanden.

\subsection{Grupperoller}
Gruppens valg om at have få roller i projektarbejdet skabte overblik, da alle i gruppen således var på lige fod. Ligeledes blev ingen defineret ud fra en titel, hvilket gjorde samarbejdet mere flydende og dynamisk. Da referentrollen ikke blev roteret, var kvaliteten af referaterne konsistent. Størstedelen af gruppen fik dog ikke træning i at tage referat. \\
Ordstyrerrollen blev roteret, hvormed alle fik mulighed for at prøve denne rolle til vejledermøder. Det var imidlertid ikke alle, der opfyldte denne rolle optimalt, hvilket gjorde nogle vejledermøder ustrukturerede. 

\subsection{Kommunikation}
Dokumentet med individuelle ambitionsniveauer var en god måde at blive bekendt med hinandens mål i begyndelsen af projektet. Det blev dog ikke anvendt til andet i løbet af projektet og mistede dermed sin funktion, da vi ikke har taget højde for hinandens forskelligheder nedskrevet i dokumentet. \\
Det var vores intention at afholde sociale arrangementer løbende, men grundet følt tidspres blev disse ofte tilsidesat til fordel for projektarbejde eller personlige planer. Vi fik imidlertid afholdt et par arrangementer, der gavnede sammenholdet og derved samarbejdet i gruppen. \\
Konflikthåndteringen fra fredagsmødets faste dagsorden viste sig at være et vigtigt punkt. I starten blev det ikke benyttet, men senere i projektet blev der drøftet problematikker, som ellers var blevet fortiet. Derved fik vi snakket om vores individuelle, og til tider personlige, problemer med hinanden indbyrdes i gruppen.

\section{Syntese}
Der er stadig delte meninger om opdelingen af samarbejdsaftalen, da de forskellige gruppemedlemmer vurderer de positive og negative sider ved opdelingen forskelligt. Nogle gruppemedlemmer foretrækker derved en mere samlet samarbejdsaftale i P$3$, hvor andre ønsker en mere opdelt og kortfattet samarbejdsaftale. \\
Det er vigtigt i P$3$ at tage samarbejsaftalen op til vurdering flere gange i projektperioden. På denne måde opdateres aftalen og alle gruppemedlemmer bliver påmindet om retningslinjerne. \\
Afholdelse af et ugentligt møde skaber overblik samt tager hånd om konflikter, hvilket vi fortsat vil benytte på P$3$. Strukturen på fredagsmødet kan imidlertid med fordel ændres, således at nogle punkter tages ved en runde frem for fri snak. Dette sikrer at alle kommer til orde i svære diskussioner. \\
Vi har været glade for vores fordeling af roller og vil i P$3$ arbejde videre på dette system. Referentrollen bør evt. roteres, og ordstyreren bør hjælpes med generelle retningslinjer for at sikre ensartede møder. \\
Der er i gruppen enighed om, at vi i P$3$ bør arbejde mere med de individuelle forventninger til projektet og hinanden. På denne måde tages hensyn til det enkelte gruppemedlems behov. Vi vil forsøge at prioritere sociale arrangementer højere end i denne projektperiode, da vi erfarede, at disse havde en positiv effekt på gruppesamarbejdet.  