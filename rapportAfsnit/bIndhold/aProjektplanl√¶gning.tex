\chapter{Projektplanlægning}

\section{Beskrivelse}

\subsection{Valg af projekt}
I forbindelse med valg af projekt udarbejdede vi hver især en prioriteringsliste ud fra projektkataloget. Derefter fandt vi i fællesskab frem til et kompromis. Projektvalget faldt på flertallets 2. prioritet.
\subsection{Tidsplanlægning}
I P2 har vi haft fire forskellige projektplaner; En overordnet tidsplan for hele projektperioden, en tavle med en tidsplan for de følgende to uger, en dagsorden for den pågældende dag samt en to-do liste nogle gange afløste dagsordnen. 
Det fælles formål for projektplanerne var at give et overblik over projektet på kort og på langt sigt.
Den overordnede tidsplan skulle give et overblik over hele projektperioden, imens tidsplanen på tavlen blev udarbejdet ud fra den overordnede plan for mere specifikt at beskrive hvad der skulle laves i løbet af de følgende to uger. Dagsordenen havde til formål at udspecificere hvad programmet for den enkelte dag var. Sidst i projektperioden blev dagsordenen dog ofte erstattet af de en to-do liste med de opgaver der manglede at blive udført for at projektet kunne blive færdiggjort.

\subsection{Projektledelse}
Vi har i projektperioden ikke haft én decideret projektleder, men i stedet anvendt hinandens kompetencer når vi har siddet med arbejdsopgaver vi ikke selv kunne løse.

\subsection{Eksperimenter}
I forbindelse med arbejde i grupperummet har vi forsøgt at arbejde ud fra 20-5 metoden, hvor vi har arbejdet koncentreret i 20 minutter og efterfølgende holdt fem minutters pause. Der var sat alarm på de 20 minutter men ikke på de fem minutters pause. 

\subsection{Rapportopsætning}
Under udarbejdelsen af projektet anvendte vi GitHub til at dele vores arbejde på. Projektet blev udarbejdet i LaTeX. 

\section{Vurdering}

\subsection{Valg af projekt}

\subsubsection{Positive erfaringer}
Vi nåede frem til et kompromis som alle var tilfredse med i forhold til hvilket projekt vi skulle arbejde med.
\subsubsection{Negative erfaringer}

\subsection{Tidsplanlægning}

\subsubsection{Positive erfaringer}
Den overordnede tidsplan fungerede godt i begyndelsen og hjalp os med at komme hurtigt igang. Tidsplanen på tavlen fungerede godt og gav et godt overblik over hvad der skulle laves i de pågældende uger. Dagsordnen gav et godt overblik over hvad der skulle laves den pågældende dag således at alle var enige herom. 

\subsubsection{Negative erfaringer}
Den overordnede tidsplan mistede sin funktion efter statusseminaret. Vi endte med at have lavet tre forskellige to-do lister hvilket var uoverskueligt.

\subsection{Projektledelse}

\subsubsection{Positive erfaringer}
I stedet for at have én person der stod med alt ansvaret udnyttede vi hinandens kompetencer indenfor forskellige områder.

\subsubsection{Negative erfaringer}
Ved nogle af de svære arbejdsopgaver var det svært at finde ud af hvem man skulle søge hjælp hos.

\subsection{Eksperimenter}

\subsubsection{Positive erfaringer}
Det fungerede godt at arbejde koncentreret i 20 minutter - vi var effektive og fik udført mange arbejdsopgaver når vi anvendte metoden.

\subsubsection{Negative erfaringer}
Planen om at holde fem minutters pause blev ofte ikke overholdt - pauserne blev tit længere.

\subsection{Rapportopsætning}

\subsubsection{Positive erfaringer}
Ved brugen af GitHub var alle gruppemedlemmer altid opdateret på hvad hinanden arbejdede med hvilket skabte et godt overblik. Det fungerede desuden godt at skrive projektet i LaTeX.

\subsubsection{Negative erfaringer}
I begyndelsen af projektperioden var det vanskeligt at anvende GitHub og LaTeX for nogle af gruppens medlemmer. 

\section{Refleksion}
