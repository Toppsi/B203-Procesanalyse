\chapter{Samarbejde med Vejleder}
I det følgende afsnit vil samarbejdet med vejleder blive beskrevet, vurderet og analyseret. 

\section{Beskrivelse}
I dette afsnit beskrives samarbejdsaftalen, som blev indgået med vejleder. Derudover vil dagsordenen samt typen af vejledning blive gennemgået.

\subsection{Samarbejdsaftale}
I starten af P$2$ indgik vi en samarbejdsaftale med vejleder, se bilag \ref{Bilag3}. Denne samarbejdsaftale var skriftlig og bestod af retningslinjer for afsendelse af dagsorden, referat af forrige vejledermøde samt materialer, som vi forventede at få feedback på. Vi aftalte derudover, at vi havde valgfrit vejledermøde hver onsdag.

\subsection{Dagsorden}
Vi udarbejdede en dagsorden til hvert vejledermøde. Dagsordenen indeholdte punkter som f.eks. update på projekt og spørgsmål, der krævede vejledning. Den blev lavet fælles på gruppen, hvor hver gruppemedlem kunne komme med spørgsmål. Der blev skrevet referat under vejledermødet af referenten, som blev sendt til vejleder efter mødet.

\subsection{Mundtlig og skriftligvejledning}
Under P$2$ har vi eksperimenteret med både mundtlig og skriftlig vejledning. Under mødet optrådte et gruppemedlem som ordstyrer, der startede mødet og uddelegerede spørgsmålene iblandt de andre medlemmer, som var tilsendt i dagsordenen. Den skriftlige vejledning omhandler vejleredens feedback på materialen, som blev sendt til hende på mail.

\section{Vurdering}
\begin{table}[h]
	\begin{tabular}{|l|p{5cm}|p{5cm|}
			\hline
			\textbf{}  & \textbf{Positive erfaringer}  & \textbf{Negative erfaringer}                                                                                   \\ \hline
			\textbf{Samarbejdsaftale}  & Samarbejdsaftale med vejleder gav enighed om samarbejde\newline Valgfrit møde hver onsdag gjorde gruppen selvstændig & Samarbejdsaftale med vejleder blev til tider ikke overholdt \\ \hline
			\textbf{Dagsorden}´& Update i mail gjorde møderne korte\newline Referat holdte gruppen opdateret & Referatet blev ikke udnyttet optimalt \\ \hline
			\textbf{Mundtlig og skriftlig vejledning} & Møde selvom sygdom via skype\newline Type af vejledning gjorde os selvstændige\newline Mødets gang  & Sprogbarriere gav til tider misforståelser\newline Uddelegering af spørgsmål under møde gav lange besvarelser \newline Uenighed i gruppen omkring optimal type vejledning      \\ \hline
		\end{tabular}
	\end{table}
	
	\section{Reflekion}
	\subsection{Samarbejdsaftale}
	Det var godt at have en samarbejdsaftale med vejleder, da dette gjorde, at der var enighed om det kommende samarbejde. Derudover er det fordelagtigt at nedskrive denne aftale i stedet for en mundtlig forventningsafstemning, da mulige uenigheder derved kan undgås. Samarbejdsaftalen blev dog ikke altid overholdt af vejleder, da e-mails ikke altid blev besvaret, samt referat fra forrige vejledermøde ikke blev gennemlæst. Vi skulle dog have gjort vejleder opmærksom på dette. \\
	Vejleder sagde fra starten, at hun havde tid til et møde hver onsdag. Dette var valgfrit, hvilket var godt, da gruppen derved selv skulle tage stilling til, om vi havde brug for vejledning.
	
	\subsection{Dagsorden}
	Vi havde gode erfaringer med at skrive en update på projektet i dagsordenen, da der derved blev sparet tid til mødets, og vejleder kunne bedre sætte sig ind i evt. spørgsmål. Derudover gjorde det både vejleder og gruppen bedre forberedt til mødet, fordi gruppen derved selv skulle tage stilling til, hvor lang vi var i processen. \\
	Referaterne fra møderne kunne bla. bruges til at informere de fraværende gruppemedlemmer om, hvad der var blevet snakket om. Derudover var hensigten med referatet, at vejleder kunne læse det efterfølgende for at sikre, at der ikke var sket nogle misforståelser. Dette blev dog ikke gjort, hvilket gruppen burde gøre vejleder opmærksom på, at vi følte var vigtigt. 
	
	\subsection{Mundtlig og skriftligvejledning}
	Selvom vores vejleder var syg, foretog hun et møde med os via Skype. Dette var godt, da vi havde brug for vejledningen, og vejleder viste god engagement i os. \\
	Gruppen er uenige om, om typen af vejledning var optimal for os. Denne type vejledning tvang os til at agere mere selvstændigt end vi var i P$1$, hvilket giver os mere erfarring til dette fremadrettet. Dog følte nogle gruppemedlemmer, at vi skulle være for selvstændige, hvilket gjorde, at vi ubevidst havde arbejdet mere emneorienteret end problemorienteret i starten af projektperioden. Vi har dog ikke givet udtryk overfor vejleder, hvilken type vejledning vi forventede. Hvis vi havde diskuteret dette med vejleder tidligere i forløbet, kunne dette muligvis være undgået. \\
	Til vejledermøder agerede et gruppemedlem som ordstyrer, hvilket indebar, at personen skulle starte mødet op. Derefter skulle ordstyrer uddele de forskellige spørgsmål fra dagsordenen til det gruppemedlem, som havde stillet spørgsmålet. Dette var godt, da personen derved selv kunne formulere sit spørgsmål ordentligt, og resten af gruppen var blevet sat ind i spørgsmålet vha. dagsordenen. Dog kunne dette spørgsmål lede ud i lange dialoger, som i sidste ende kunne være langt fra det oprindelige emne og ikke besidde interesse fra resten af gruppen. \\
	Da vores vejleders modersmål ikke er dansk, har dette til tider givet nogle misforståelser imellem gruppen og vejleder. Vejleder sagde dog fra starten af, at vi selv måtte bestemme, om vi ville snakke på dansk eller engelsk.
	
	\section{Syntese}
	I P$3$ vil vi fortsat udarbejde en skriftlig samarbejdsaftale med vejleder. Hvis aftalen ikke bliver overholdt, vil det dog være fordelagtigt at gøre vejleder opmærksom på dette hurtigst muligt. Vi vil fortsat skrive et referat ved hvert vejledermøde, da dette både er godt for gruppen og vejleder. Derudover vil det være optimalt, hvis vejleder igen har til et valgfrit møde en gang om ugen, samt vejleder og gruppen bliver enige om, hvilket type vejledning, der ønskes. I tilfælde af at vores vejleders modersmål til næste projekt igen ikke er dansk, så vil gruppen fra starten være mere opmærksom på, hvilket sprog vejleder foretrækker for at undgå misforståelser.