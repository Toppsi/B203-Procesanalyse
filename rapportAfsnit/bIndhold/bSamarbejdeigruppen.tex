\chapter{Samarbejde i Gruppen}
For at sikre et godt samarbejde i gruppen er det nødvendigt for gruppen at blive enige om, hvordan arbejdet skal forløbe. Dette kræver at gruppen forventningsafstemmer, danner en samarbejdsaftale og er opmærksom på at have god kommunikation. Disse vil i de følgende afsnit blive beskrevet, reflekteret og analyseret.

\section{Beskrivelse}
I dette afsnit vil de enkelte faktorer for gruppens samarbejde med hinanden blive beskrevet. 

\subsection{Samarbejdsaftale}
I starten af projektperioden udarbejdede gruppen en samarbejdsaftale. Der var uenighed om, hvad en samarbejdsaftale skulle indeholde og hvordan den skulle sættes op. Derfor lavede gruppen både en samarbejdsaftale og et dokument kaldet "Tips og Tricks". Disse er vedlagt i bilag \ref{Bilag1} og bilag \ref{bilag2}. Samarbejdsaftalen var krav og forventninger til gruppens medlemmer, hvor dokumentet "Tips og Tricks" omhandlede retningslinjer og forslag til arbejdsformer og metoder, som gruppen kunne gøre brug af. Samarbejdsaftalen var fast og blev ikke ændret igennem hele projekforløbet, hvorimod "Tips og Tricks" blev omskrevet i takt med at gruppen fandt frem til mere optimale arbejdsmetoder. 

\subsection{Gruppemøder og Gruppearbejde}
Gruppen vedtog i starten af projektperioden at vi skulle mødes hver dag fra kl. $9.15$ til $16.15$, medmindre andet var aftalt. Sluttiden varierede i begyndelsen af projektperioden og blev sat individuelt. Senere i projektet blev mødetiden ændret til $8.15$ og den fastsatte sluttid blev overholdt. Der var i starten ingen aftale om fremmøde til semestrets tilhørende kurser, men det blev senere vedtaget at det var obligatorisk for gruppens medlemmer at møde op til kursernes opgaveregning.
Gruppen har derudover gjort brug af tre forskellige møder; fredagsmøder, vejledermøder og statusmøder.

Fredagsmøder blev afholdt hver fredag og havde til formål at opsummere ugens forgangne arbejde, overholdelse af tidsplan og konflikthåndtering. Til disse møder var der nedskrevet en fast skabelon for dagsorden, der kan ses i bilag \ref{Bilag4}.

Vejledermøder var et møde mellem gruppen og vejleder, og bliver gennemgået yderligere i \fxnote{Indsæt ref}

Statusmøder var et kort møde med en fast dagsorden, der blev afholdt ved starten og slutningen af hver dag. Dagsordenen for dette møde indbefattede en runde, hvor hvert gruppemedlem fortalte egen status for det individuelle arbejde.

\subsection{Grupperoller}
Gruppen gjorde brug af to roller ved møder og gruppearbejde; ordstyrer og referent. Ordstyrer styrede samtalen til møder, stoppede ligegyldige
eller langtrukne diskussioner og kaldte til fælles pause. Referanten tog referat ved møder og skrev dagsorden for møderne. Gruppen forsøgte at undgå for mange forskellige roller.

\subsection{Kommunikation}
Gruppen startede ud med at snakke om vores individuelle ambitionsniveau og forventninger for dette projekt. Der blev udarbejdet nogle punkter, som hvert gruppemedlem skulle komme ind på. Alt der blev nævnt, blev skrevet ned i et fælles dokument, som alle gruppemedlemmer havde tilgang til. Derudover blev gruppen enige om at prioritere sociale arrangementer med hinanden og hver fredag havde gruppen et punkt på dagsordenen, der indbefattede konflikthåndtering. 
