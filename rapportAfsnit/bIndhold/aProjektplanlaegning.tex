\chapter{Projektplanlægning}
I det følgende afsnit vil projektplanlægning blive beskrevet, vurderet og analyseret. 

\section{Beskrivelse}
I dette afsnit vil valget af projekt, tidsplanlæsning, projektledelse, eksperimenter og rapportopsætning blive beskrevet.

\subsection{Valg af projekt}
I forbindelse med valg af projekt udarbejdede vi hver især en prioriteringsliste ud fra projektkataloget. Derefter fandt vi i fællesskab frem til et kompromis. Projektvalget faldt på flertallets $2$. prioritet.

\subsection{Tidsplanlægning}
I P$2$ havde vi fire forskellige projektplaner; En overordnet tidsplan for hele projektperioden, en tavle med en tidsplan for de følgende to uger, en dagsorden for den pågældende dag samt en to-do liste der nogle gange afløste dagsordenen. 
Det fælles formål for projektplanerne var at give et overblik over projektet på kort og lang sigt.
Den overordnede tidsplan skulle give et overblik over hele projektperioden, imens tidsplanen på tavlen blev udarbejdet ud fra den overordnede plan. Dagsordenen havde til formål at udspecificere, hvad programmet var for den enkelte dag. Sidst i projektperioden blev dagsordenen dog ofte erstattet af en to-do liste med de opgaver, der manglede at blive udført.

\subsection{Projektledelse}
Vi havde i projektperioden ikke en decideret projektleder men anvendte i stedet hinandens kompetencer, når vi sad med arbejdsopgaver, som vi ikke selv kunne løse.

\subsection{Eksperimenter}
I forbindelse med arbejde i grupperummet forsøgte vi at arbejde ud fra $20$-$5$ metoden, hvor vi arbejdede koncentreret i $20$ minutter og efterfølgende holdt fem minutters pause. Der var sat alarm på de $20$ minutter men ikke efter de fem minutters pause. 

\subsection{Rapportopsætning}
Under udarbejdelsen af projektet anvendte vi GitHub til at dele vores arbejde på. Projektet blev udarbejdet i LaTeX. 

\section{Vurdering}
Her opstilles de beskrevne emner i positive og negative erfaringer
%\subsection{Valg af projekt}


\begin{table}[h]
	%\caption{This Is a Table\label{LABEL}}
	\begin{tabular}{|l|p{5cm}|p{5cm}|}
		\hline
		&\textbf{Positive erfaringer}&\textbf{Negative erfaringer}\\ \hline
		Valg af projekt  & Vi nåede frem til et kompromis.& \\ \hline  
		Tidsplanlægning  & Overordnet tidsplan: Hjalp os med at komme hurtigt igang.\newline
		Ugenplan: Gav et bredere overblik.\newline
		Dagsorden: Gav et kortsigtet detaljeret overblik.\newline
		To-do liste: Effektiviserede arbejdet. & Overordnet tidsplan: Var vanskelig at overholde. \newline 
		To-do liste: Ustruktureret. \\ \hline
		Projektledelse   & Forskellige gruppemedlemmers kompetencer blev udnyttet. & Til tider vanskeligt at opsøge hjælp. \\ \hline
		Eksperimenter    & $20$-$5$ metoden effektiviserede arbejdet. & Pauserne blev ofte for lange. \\ \hline
		Rapportopsætning & GitHub: Alle var opdateret på hinandens arbejde.\newline
		LaTeX: Gjorde skriveprocessen overskuelig. & Stejl læringskurve for læring af LaTex.\\ \hline
     
	\end{tabular}
\end{table}


%\subsubsection{Positive erfaringer}
%Vi nåede frem til et kompromis som alle var tilfredse med i forhold til hvilket projekt vi skulle arbejde med.
%\subsubsection{Negative erfaringer}

%\subsection{Tidsplanlægning}
%
%\subsubsection{Positive erfaringer}
%Den overordnede tidsplan fungerede godt i begyndelsen og hjalp os med at komme hurtigt igang. Tidsplanen på tavlen fungerede godt og gav et godt overblik over hvad der skulle laves i de pågældende uger. Dagsordnen gav et godt overblik over hvad der skulle laves den pågældende dag således at alle var enige herom. Vi arbejdede effektivt ud fra to-do listen. 
%
%\subsubsection{Negative erfaringer}
%Den overordnede tidsplan mistede sin funktion efter statusseminaret. Vi endte med at have lavet tre forskellige to-do lister hvilket var uoverskueligt.
%
%\subsection{Projektledelse}
%
%\subsubsection{Positive erfaringer}
%I stedet for at have én person der stod med alt ansvaret udnyttede vi hinandens kompetencer indenfor forskellige områder.
%
%\subsubsection{Negative erfaringer}
%Ved nogle af de svære arbejdsopgaver var det svært at finde ud af hvem man skulle søge hjælp hos.
%
%\subsection{Eksperimenter}
%
%\subsubsection{Positive erfaringer}
%Det fungerede godt at arbejde koncentreret i 20 minutter - vi var effektive og fik udført mange arbejdsopgaver når vi anvendte metoden.
%
%\subsubsection{Negative erfaringer}
%Planen om at holde fem minutters pause blev ofte ikke overholdt - pauserne blev tit længere.

%\subsection{Rapportopsætning}
%
%\subsubsection{Positive erfaringer}
%Ved brugen af GitHub var alle gruppemedlemmer altid opdateret på hvad hinanden arbejdede med hvilket skabte et godt overblik. Det fungerede desuden godt at skrive projektet i LaTeX.
%
%\subsubsection{Negative erfaringer}
%I begyndelsen af projektperioden var det vanskeligt at anvende GitHub og LaTeX for nogle af gruppens medlemmer. 

\section{Refleksion}
I dette afsnit reflekteres de førnævnte emner.

\subsection{Valg af Projekt}
Vi nåede frem til et tilfredsstillende kompromis i forhold til valg af projekt, da vi på skift snakkede om vores præferencer og derefter kom frem til enighed. Vi havde alle skrevet en prioriteringsliste og var på den måde klar til at komme med argumenter for, hvorfor det ville være mest interessant at arbejde med de projekter, vi hver især havde valgt. På den måde fik vi allerede første dag i projektperioden en god diskussion og kom omkring stort set alle projekterne i kataloget, inden vi kunne træffe den endelige beslutning. 

\subsection{Tidsplanlægning}
Den overordnede tidsplan fungerede godt i begyndelsen, da oversigten gav os et overblik for, hvad vi skulle i gang med. Efterfølgende fungerede den også, men da vi nåede til statusseminaret, blev vi bevidste om, at en stor del af projektet burde omstruktureres. Derefter holdt den overordnede plan ikke længere, og tidsplanen mistede dermed sin funktion i den resterende del af projektet. 
Tidsplanen på tavlen gav et godt overblik over projektarbejdet, da der overordnet var beskrevet, hvad der skulle laves i løbet af de følgende to uger. Desuden var kurserne skrevet ind i tidsplanen, hvilket gjorde det lettere for os at overskue, hvor meget tid vi overordnet havde til projektarbejde i de pågældende uger. 
Dagsordenen for dagens arbejde blev udarbejdet på baggrund af tidsplanen på tavlen og var på den måde medvirkende til at udspecificere, hvad der skulle laves den pågældende dag. Derved havde vi overblik for, hvad vi skulle nå, og det skabte struktur på de enkelte arbejdsdage.
To-do listerne var gode til at give et overblik over, hvilke arbejdsopgaver der manglede for at projektet kunne blive færdiggjort. Listen var derved medvirkende til, at vi arbejdede effektivt. Vi endte dog ud med at have tre forskellige to-do lister, hvilket ikke var godt og skabte uorden. Vi mistede på den måde overblikket af og til over, hvad vi egentlig manglede, og om arbejdsopgaverne på den foregående to-do liste allerede var færdiggjort.

\subsection{Projektledelse}
Det har fungeret godt i vores gruppe at være foruden en projektleder, da vi overordnet set hver især har haft nogle gode kompetencer inden for forskellige områder. Derfor har der ikke været en person i gruppen, der kunne hjælpe med alle arbejdsopgaverne. Vi har i stedet haft muligheden for at vurdere, hvem i gruppen det har været mest optimalt at søge hjælp hos ud fra den opgave, som man har siddet med. Dette var dog også en ulempe, da der ved nogle af de svære arbejdsopgaver ikke var en person i gruppen, der havde en særlig viden indenfor området. I disse situationer var det vanskeligt at vurdere, hvem man skulle søge hjælp hos, da der ikke var en leder, som man altid kunne gå direkte til. I de fleste tilfælde blev disse problemer dog løst på et gruppemøde, hvor hele gruppen hjalp med at nå frem til en måde, hvorpå arbejdsopgaven kunne løses.

\subsection{Eksperimenter}
Det fungerede godt at arbejde koncentreret i $20$ minutter ad gangen, da dette var et overskueligt tidsrum at holde fuld koncentration i. På den måde undgik man at blive ukoncentreret og komme til at forstyrre de øvrige gruppemedlemmer i deres arbejde. Det fastsatte tidsrum var også medvirkende til, at man kunne sætte sig personlige mål for, hvor meget man skulle nå at have lavet, før der igen var pause. På den måde blev arbejdet mere effektivt. 
Pausen, der burde vare fem minutter, blev ofte længere, hvilket formentlig skyldtes, at der ikke blev sat en alarm til påmindelse af afsluttet pause. Dermed opstod en tendens til, at vi glemte tiden i pauserne. Dette var ikke optimalt, da vi kunne have brugt den ekstra pausetid på at arbejde i stedet.

\subsection{Rapportopsætning}
Anvendelsen af GitHub gav et godt overblik over, hvad de enkelte gruppemedlemmer arbejdede på. Grunden til dette var, at man skulle gemme sit arbejde i en fællesmappe og derefter skrive en kort beskrivelse af, hvad man havde lavet på Github. 
I begyndelsen af projektperioden var vanskeligt for nogle gruppemedlemmer at anvende GitHub og LaTeX, da det var programmer, som vi ikke havde anvendt før. Dette gav nogle problemer på GitHub, da filerne nogen gange ikke blev tilføjet korrekt til fællesmappen. Dette var dog ikke længere et problem i samme omfang, da vi kom længere ind i projektperioden og alle fik bedre styr på programmerne. 

\section{Syntese}
Hvis valg af projekt i P$3$ bliver vanskeligt, vil vi anvende samme metode, som vi har anvendt i P$2$. Derved indgår alle gruppemedlemmer i en dialog og kan argumentere for deres prioriteter. 
Vi vil i P$3$ forsøge at revidere den overordnede tidsplan oftere, således at den altid er opdateret og dermed anvendes mere. Revisionen kan eventuelt tilføjes som et fast punkt på et ugentligt møde. Ugeplanen på tavlen er fortsat et godt supplement til den overordnede tidsplan, såvel som dagsorden og en to-do liste. Strukturen bag dagsorden og to-do liste bør dog på P$3$ klarlægges fra projektstarten, således at de ikke forvirrer folk mere end de gavner.
Der er enighed i gruppen om, at der på P$3$ ikke bør være en projektleder, da holdningen er, at en projektleder vil bryde gruppedynamikken.
Vi vil i P$3$ anvende $20$-$5$ metoden fra projektperiodens start, da vi i P$2$ erfarede, at vi arbejdede mere effektivt ved brug af denne metode. Vi vil dog være mere konsekvente i forhold til at overholde pausetiden, således at vi ikke holder for lange pauser. 
Udarbejdelse og deling af projektrapporten i LaTeX og GitHub vil vi i P$3$ arbejde videre med. Disse programmer gør de individuelle arbejdsopgaver overskuelige for de øvrige gruppemedlemmer og danner et samlet overblik over projektet.   