\section{Syntese}
Der er stadig delte meninger om opdelingen af samarbejdsaftalen, da de forskellige gruppemedlemmer vurderer de positive og negative sider ved opdelingen forskelligt. Nogle gruppemedlemmer foretrækker derved en mere samlet samarbejdsaftale i P$3$, hvor andre ønsker en mere opdelt og kortfattet samarbejdsaftale. \\
Det er vigtigt i P$3$ at tage samarbejsaftalen op til vurdering flere gange i projektperioden. På denne måde opdateres aftalen og alle gruppemedlemmer bliver påmindet om retningslinjerne. \\
Afholdelse af et ugentligt møde skaber overblik samt tager hånd om konflikter, hvilket vi fortsat vil benytte på P$3$. Strukturen på fredagsmødet kan imidlertid med fordel ændres, således at nogle punkter tages ved en runde frem for fri snak. Dette sikrer at alle kommer til orde i svære diskussioner. \\
Vi har været glade for vores fordeling af roller og vil i P$3$ arbejde videre på dette system. Referentrollen bør evt. roteres, og ordstyreren bør hjælpes med generelle retningslinjer for at sikre ensartede møder. \\
Der er i gruppen enighed om, at vi i P$3$ bør arbejde mere med de individuelle forventninger til projektet og hinanden. På denne måde tages hensyn til det enkelte gruppemedlems behov. Vi vil forsøge at prioritere sociale arrangementer højere end i denne projektperiode, da vi erfarede, at disse havde en positiv effekt på gruppesamarbejdet.  