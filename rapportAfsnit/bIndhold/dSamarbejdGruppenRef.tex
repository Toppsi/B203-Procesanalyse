\section{Refleksion}
I dette afsnit vil de enkelte faktorer for gruppens samarbejde med hinanden blive reflekteret.

\subsection{Samarbejdsaftale}
At have samarbejdsaftalen på skrift var positivt, da det gjorde aftalen permanent og alle gruppemedlemmer tog derved aftalen mere seriøst. Derudover gav det enighed om, hvilke retningslinjer gruppearbejdet skulle forløbe under. Dette førte til et bedre samarbejde, da alle gruppemedlemmer var enige om og indforstået med kriterierne for et godt samarbejde.

Opdelingen af samarbejdsaftalen var et kompromis i gruppen og havde både positive og negative konsekvenser. Selve samarbejdsaftalen blev kort og præcis, hvorved den blev lettere at overholde. Dokumentet "Tips og Tricks" blev dog taget mindre serisøst end samarbejdsaftalen og opdelingen medførte derudover at aftalen som helhed blev sværre at overskue. Disse faktorer medførte at "Tips og Tricks" ikke blev  overholdt og håndhævet af gruppen.

\subsection{Gruppemøder og Gruppearbejde}
Den faste mødetid gav overblik over samarbejdet, da det var muligt for alle gruppemedlemmer at følge med i hinandens arbejde. Dette gav tryghed for nogle gruppemedlemmer, da det var muligt at se projektets progressive forløb. Det er dog vanskeligt at holde fuld koncentration hele dagen, hvilket medførte et splittet og opdelt arbejdsflow.

Opgaveregningen for semestres tilhørende kurser blev gjort obligatorisk for alle gruppemedlemmer, da fremmødet hertil var mangelfuldt. Det lille fremmøde gjorde det vanskeligt at løse opagverne, hvormed gruppen blev enige om at gøre opgaveregningen obligatorisk. Hermed tog gruppen et fælles ansvar for indlærring, hvilket udover at forbedre forståelsen for kurserne også gav bedre sammenhold i gruppen. Denne regel tog dog ikke hensyn til gruppemedlemmernes forskellige foretrukne arbejdsmetoder, hvilket kan have hæmmet indlæringen ved nogle af os. Gruppen forsøgte at tage højde for dette, da fremmøde til forelæsninger ikke blev gjort obligatorisk, men bare den tilhørende opgaveregning.

Den faste dagsorden for fredagsmøderne sikrede at gruppen kom igennem alle aftalte punkter på hvert fredagsmøde. Dette gav god mødestruktur og sikrede at gruppen ikke sprang vigtige emner over. Derudover var fredagsmøderne med til at give overblik over status for projektets, da vi her gennemgik den planlagte tidsplan. Hermed havde gruppen et fælles og ensartet overblik over projektet, hvilket er en god forudsætning for et godt samarbejde. 

Konflikthåndteringen på fredagsmødet var et godt tiltag for gruppen. Det medførte at vi fik snakket om problemer og misforståelser, hvilket styrkede sammenholdet i gruppen. Gruppen var enige om at konflikthåndteringen skulle foregå konstruktivt, hvorved der ikke foregik tilsvining af hinanden. Da konflikthåndtering var et punkt på dagsordenen, blev der iblandt taget ting op til diskussion, som gruppemedlemmer ellers ville have tiet. Dette medførte et tættere samarbejde, da der ikke var små irritationer mod hinanden, som forblev usagt. 


